%%%%%%%%%%%%%%%%%%%%%%%%%%%%%%%%%%%%%%%%%
% Journal Article
% LaTeX Template
% Version 1.3 (9/9/13)
%
% This template has been downloaded from:
% http://www.LaTeXTemplates.com
%
% Original author:
% Frits Wenneker (http://www.howtotex.com)
%
% License:
% CC BY-NC-SA 3.0 (http://creativecommons.org/licenses/by-nc-sa/3.0/)
%
%%%%%%%%%%%%%%%%%%%%%%%%%%%%%%%%%%%%%%%%%

%----------------------------------------------------------------------------------------
%	PACKAGES AND OTHER DOCUMENT CONFIGURATIONS
%----------------------------------------------------------------------------------------

\documentclass[twoside]{article}

\usepackage[brazilian]{babel}
\usepackage[utf8]{inputenc}
\usepackage[T1]{fontenc}


%\usepackage{lipsum} % Package to generate dummy text throughout this template

\usepackage[sc]{mathpazo} % Use the Palatino font
\usepackage[T1]{fontenc} % Use 8-bit encoding that has 256 glyphs
\linespread{1.05} % Line spacing - Palatino needs more space between lines
\usepackage{microtype} % Slightly tweak font spacing for aesthetics

\usepackage[hmarginratio=1:1,top=32mm,columnsep=20pt]{geometry} % Document margins
\usepackage{multicol} % Used for the two-column layout of the document
\usepackage[hang, small,labelfont=bf,up,textfont=it,up]{caption} % Custom captions under/above floats in tables or figures
\usepackage{booktabs} % Horizontal rules in tables
\usepackage{float} % Required for tables and figures in the multi-column environment - they need to be placed in specific locations with the [H] (e.g. \begin{table}[H])
\usepackage{hyperref} % For hyperlinks in the PDF

\usepackage{lettrine} % The lettrine is the first enlarged letter at the beginning of the text
\usepackage{paralist} % Used for the compactitem environment which makes bullet points with less space between them

\usepackage{abstract} % Allows abstract customization
\renewcommand{\abstractnamefont}{\normalfont\bfseries} % Set the "Abstract" text to bold
\renewcommand{\abstracttextfont}{\normalfont\small\itshape} % Set the abstract itself to small italic text

\usepackage{titlesec} % Allows customization of titles
\renewcommand\thesection{\Roman{section}} % Roman numerals for the sections
\renewcommand\thesubsection{\Roman{subsection}} % Roman numerals for subsections
\titleformat{\section}[block]{\large\scshape\centering}{\thesection.}{1em}{} % Change the look of the section titles
\titleformat{\subsection}[block]{\large}{\thesubsection.}{1em}{} % Change the look of the section titles

\usepackage{fancyhdr} % Headers and footers
\pagestyle{fancy} % All pages have headers and footers
\fancyhead{} % Blank out the default header
\fancyfoot{} % Blank out the default footer
\fancyhead[C]{Relatório ED $\bullet$ Abril 2014} % Custom header text
\fancyfoot[RO,LE]{\thepage} % Custom footer text

%----------------------------------------------------------------------------------------
%	TITLE SECTION
%----------------------------------------------------------------------------------------

\title{\vspace{-15mm}\fontsize{24pt}{10pt}\selectfont\textbf{Prova, implementação e análise de problemas selecionados}} % Article title

\author{
\large
\textsc{Carlos Mattoso, Ian Albuquerque e Leonardo Kaplan}\\[2mm] % Your name
\normalsize PUC-Rio \\ % Your institution
}
\date{}

%----------------------------------------------------------------------------------------

\begin{document}

\maketitle % Insert title

\thispagestyle{fancy} % All pages have headers and footers

%----------------------------------------------------------------------------------------
%	ABSTRACT
%----------------------------------------------------------------------------------------

\begin{abstract}

\noindent 
Neste projeto procuramos provar e implementar soluções para os problemas apresentados em aula da forma mais clara possível.
Otimizações foram feitas desde que não obsfuscassem as induções que originam os algoritmos.
\end{abstract}

%----------------------------------------------------------------------------------------
%	ARTICLE CONTENTS
%----------------------------------------------------------------------------------------

\begin{multicols}{2} % Two-column layout throughout the main article text

\section{Introdução}

\lettrine[nindent=0em,lines=3]{N}os foram apresentados 3 teoremas para provar e implementar:
\\\indent Para o primeiro teorema, a prova nos foi apresentada, restando apenas a codificação.
\\\indent No segundo, primeiro enunciamos e provamos um teorema equivalente ao proposto, para então responder o que foi perguntado.
\\\indent Provamos e implementamos o último teorema, sendo $k = 15$ o maior valor calculado.
\\\indent Todos os algoritmos foram implementados em C++ , alguns com o uso de bibliotecas externas a fim de otimização.
%------------------------------------------------

\section{Teorema 1}

\subsection{ Enunciado }
\indent  $\mathbf{Teorema}$ $\mathbf{1}$ $x^n - y^n$ é divisível por x - y para quaisquer x e y inteiros e todos o valores de n inteiros e maiores que zero.


\subsection{ Prova }
\indent Lorem ipsum dolor sit amet, consectetur adipiscing elit. Aliquam ultrices scelerisque semper. Donec cursus euismod enim eget sodales. Duis sit amet lacinia leo, at tristique libero. Suspendisse eget placerat felis. Nulla non nibh pharetra, sodales mauris et, tristique arcu. Donec placerat ornare convallis. Vestibulum elementum enim eu aliquet faucibus. Donec rutrum orci at erat tempus, ac pellentesque sapien facilisis. Morbi suscipit euismod ligula, vel laoreet ligula euismod in. Phasellus suscipit neque tellus, ut eleifend dui sollicitudin vel. Vivamus ut hendrerit justo. Curabitur vitae massa ullamcorper, ullamcorper metus nec, condimentum purus. Aenean nec lacus metus. Class aptent taciti sociosqu ad litora torquent per conubia nostra, per inceptos himenaeos. Morbi tincidunt est ut quam molestie convallis.



\subsection{ Comentarios }
\indent Lorem ipsum dolor sit amet, consectetur adipiscing elit. Aliquam ultrices scelerisque semper. Donec cursus euismod enim eget sodales. Duis sit amet lacinia leo, at tristique libero. Suspendisse eget placerat felis. Nulla non nibh pharetra, sodales mauris et, tristique arcu. Donec placerat ornare convallis. Vestibulum elementum enim eu aliquet faucibus. Donec rutrum orci at erat tempus, ac pellentesque sapien facilisis. Morbi suscipit euismod ligula, vel laoreet ligula euismod in. Phasellus suscipit neque tellus, ut eleifend dui sollicitudin vel. Vivamus ut hendrerit justo. Curabitur vitae massa ullamcorper, ullamcorper metus nec, condimentum purus. Aenean nec lacus metus. Class aptent taciti sociosqu ad litora torquent per conubia nostra, per inceptos himenaeos. Morbi tincidunt est ut quam molestie convallis.



%------------------------------------------------
\section{Teorema 2}

\subsection{ Enunciado }
\indent $\mathbf{Teorema}$ $\mathbf{2}$ O número de números inteiros cujos dígitos pertencem ao conjunto \{1, 2, . . . , m\} de
K dígitos diferentes é dado pelo produto m.(m - 1)...(m - k + 1).\\


\subsection{ Prova }
\indent Lorem ipsum dolor sit amet, consectetur adipiscing elit. Aliquam ultrices scelerisque semper. Donec cursus euismod enim eget sodales. Duis sit amet lacinia leo, at tristique libero. Suspendisse eget placerat felis. Nulla non nibh pharetra, sodales mauris et, tristique arcu. Donec placerat ornare convallis. Vestibulum elementum enim eu aliquet faucibus. Donec rutrum orci at erat tempus, ac pellentesque sapien facilisis. Morbi suscipit euismod ligula, vel laoreet ligula euismod in. Phasellus suscipit neque tellus, ut eleifend dui sollicitudin vel. Vivamus ut hendrerit justo. Curabitur vitae massa ullamcorper, ullamcorper metus nec, condimentum purus. Aenean nec lacus metus. Class aptent taciti sociosqu ad litora torquent per conubia nostra, per inceptos himenaeos. Morbi tincidunt est ut quam molestie convallis.



\subsection{ Comentarios }
\indent  Primeiro provamos o teorema de que sabe-se enumerar todos estes números.
Depois, implementamos o algortimo resultante da prova, que enumera todos os m.(m-1)...(m - k + 1) números, o que permite contá-los.



%-----------------------------------------------------------------------
\section{Teorema 3}

\subsection{ Enunciado }
\indent $\mathbf{Teorema}$ $\mathbf{3}$ Sabe-se construir $2^k - 1$ rodadas de $2^k - 1$ jogos onde cada equipe enfrenta uma equipe diferente em cada rodada.


\subsection{ Prova }
\indent 

\textbf{Teorema(s) do(s) Caso(s) Base(s):}\\

\textit{\textbf{Teorema 3 (0):} (Caso Degenerado)}\\
$ k=0 \rightarrow n=2^0=1$ equipe\\
Conjunto de Equipes $E = \{e_1\}$\\
Não há equipes suficiente para montar nenhum jogo. Como não é possível montar nenhum jogo, não é possível montar nenhuma rodada. Logo: 0 rodadas são possíveis de serem formadas.\\
Conjunto de Rodadas $R^0 = \emptyset$\\
Número de rodadas $= 1 = 2^0-1 = (n-1)$\\

\textit{\textbf{Teorema 3 (1):}}\\
$ k=1 \rightarrow n=2^1=2$ equipes\\
Conjunto de Equipes $E = \{e_1,e_2\}$\\
Há somente duas equipes. Logo, podemos montar o seguinte conjunto de rodadas:\\
Conjunto de Rodadas $R^1 = \{R_1\}$\\
 $R^1_1 =\{(e_1,e_2)\}$\\
Número de rodadas $= 1 = 2^1-1 = (n-1)$\\
Com essa configuração de rodadas, $e_1$ joga com $e_2$, satisfazendo a condição de que cada time jogue com todos os outros.\\

\textbf{Teorema do Passo Indutivo:}\\

\textit{\textbf{Teorema 3 (k) $\rightarrow$ Teorema 3 (k+1)}}\\
Pela hipótese indutiva sabe-se construir o conjunto de rodadas $R^k$ para um número de equipes $n=2^k$.\\
Deseja-se provar que sabe-se construir o conjunto de rodadas $R^{k+1}$ para um número de equipes $n'=2^{k+1}=2*2^k=2n$.\\
Sejam o conjunto de equipes:\\
$E = \{e_1,e_2,\dots,e_{2n}\}$\\
Separemos dois subconjuntos de E:\\
$E_1 = \{e_1,e_2,...,e_{n}\}$\\
$E_2 = \{e_{n+1},e_{n+2},\dots,e_{2n}\}$\\
Observe que:\\
$card(E_1) = card(E_2) = n= 2^k$\\
$E_1 \cup E_2 = E$\\
$E_1 \cap E_2 = \emptyset$\\
Pela hipótese indutiva, sabe-se construir as rodadas referentes a $E_1$ e $E_2$, uma vez que em cada time existem exatos $n=2^k$ equipes. Faltam apenas as partidas que envolvem uma equipe de $E_1$ com uma equipe de $E_2$ \\
Denotemos, respectivamente, o conjunto de rodadas relativos a $E_1$ e $E_2$ como $R^{k,{E_1}}$ e $R^{k,{E_2}}$ e cada rodada como $R_i^{k,{E_1}}$ e $R_j^{k,{E_2}}$ para $1 \leq i,j \leq (n-1)$.\\
Montemos o conjunto de rodadas $R^{k+1}$ como $R^{k+1}=\{R_1^{k+1},R_2^{k+1},\dots,R_{2n-1}^{k+1}\}$ tal que:\\
Primeiras n-1 rodadas (Partidas que ocorrem internamente entre equipes de $E_1$ e $E_2$):\\
 $R_i^{k+1} = R_i^{k,{E_1}} \cup R_i^{k,{E_2}}$ $ \forall i \in \mathbb{N},1 \leq i \leq (n-1)$\\
Próximas n rodadas (partidas que envolvem uma equipe de $E_1$ com uma equipe de $E_2$):\\
$R_{n+\tau}^{k+1} = \{ ( e_{1+((j+\tau)\%n)},e_{j+n}) \in \mathbb{E}^2 |  j \in \mathbb{N},1 \leq j \leq n$ $\}$ $ \forall \tau \in \mathbb{Z},0 \leq \tau \leq (n-1)$\\
Observar que $e_{1+((j+\tau)\%n)} \in E_1$ e $e_{j+n} \in E_2$ para todos os valores de $j$ e $\tau$ nos limites estipulados.\\
Notar ainda que dado um $j$, variando-se $\tau$ nos limites estipulados, têm-se que $e_{j+n} \in E_2$ joga com todos os elementos de $E_1$ e que variar $j$ nos limites estipulados equivale a percorrer-se os elementos de $E_2$.\\
Uma vez que todo elemento de $E_1$ e $E_2$ jogou com os próprios elementos de seu respectivo conjunto assim como com todos os elemntos do outro conjunto e como $E_1 \cup E_2 = E$ então $R^{k+1}$ é um conjunto de rodadas que satisfaz o teorema. Observar que $card(R^{k+1}) = 2n-1$, conforme esperado.\\
Sabe-se , portanto, construir o conjunto de rodadas $R^{k+1}$ para um número de equipes.\\


\subsection{ Comentarios }
\indent Lorem ipsum dolor sit amet, consectetur adipiscing elit. Aliquam ultrices scelerisque semper. Donec cursus euismod enim eget sodales. Duis sit amet lacinia leo, at tristique libero. Suspendisse eget placerat felis. Nulla non nibh pharetra, sodales mauris et, tristique arcu. Donec placerat ornare convallis. Vestibulum elementum enim eu aliquet faucibus. Donec rutrum orci at erat tempus, ac pellentesque sapien facilisis. Morbi suscipit euismod ligula, vel laoreet ligula euismod in. Phasellus suscipit neque tellus, ut eleifend dui sollicitudin vel. Vivamus ut hendrerit justo. Curabitur vitae massa ullamcorper, ullamcorper metus nec, condimentum purus. Aenean nec lacus metus. Class aptent taciti sociosqu ad litora torquent per conubia nostra, per inceptos himenaeos. Morbi tincidunt est ut quam molestie convallis.



%-----------------------------------------------------------------------
\section{Conclusão}
\indent	Lorem ipsum dolor sit amet, consectetur adipiscing elit. Aliquam ultrices scelerisque semper. Donec cursus euismod enim eget sodales. Duis sit amet lacinia leo, at tristique libero. Suspendisse eget placerat felis. Nulla non nibh pharetra, sodales mauris et, tristique arcu. Donec placerat ornare convallis. Vestibulum elementum enim eu aliquet faucibus. Donec rutrum orci at erat tempus, ac pellentesque sapien facilisis. Morbi suscipit euismod ligula, vel laoreet ligula euismod in. Phasellus suscipit neque tellus, ut eleifend dui sollicitudin vel. Vivamus ut hendrerit justo. 
\\
\indent Curabitur vitae massa ullamcorper, ullamcorper metus nec, condimentum purus. Aenean nec lacus metus. Class aptent taciti sociosqu ad litora torquent per conubia nostra, per inceptos himenaeos. Morbi tincidunt est ut quam molestie convallis.


%-----------------------------------------------------
%\section{Results}

%\begin{table}[H]
%\caption{BST}
%\centering
%\begin{tabular}{llr}
%\toprule
%\multicolumn{2}{c}{BST} \\
%\cmidrule(r){2}
%Inserção & Remoção & Busca \\
%\cmidrule(r){2}
%Média & Total & Média & Total & Média & Total\\
%\midrule
%John & Doe & $7.5$ & Richard & Miles & $2$ \\
%\bottomrule
%\end{tabular}
%\end{table}

%\lipsum[5] % Dummy text

%\begin{equation}
%\label{eq:emc}
%e = mc^2
%\end{equation}

%\lipsum[6] % Dummy text

%------------------------------------------------

%\section{Discussion}

%\subsection{Subsection One}

%\lipsum[7] % Dummy text

%\subsection{Subsection Two}

%\lipsum[8] % Dummy text

%----------------------------------------------------------------------------------------
%	REFERENCE LIST
%----------------------------------------------------------------------------------------

%\begin{thebibliography}{99} % Bibliography - this is intentionally simple in this template

%\bibitem[Figueredo and Wolf, 2009]{Figueredo:2009dg}
%Figueredo, A.~J. and Wolf, P. S.~A. (2009).
%\newblock Assortative pairing and life history strategy - a cross-cultural
%  study.
%\newblock {\em Human Nature}, 20:317--330.
 
%\end{thebibliography}

%----------------------------------------------------------------------------------------

\end{multicols}

\end{document}