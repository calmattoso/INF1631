%%%%%%%%%%%%%%%%%%%%%%%%%%%%%%%%%%%%%%%%%
% Journal Article
% LaTeX Template
% Version 1.3 (9/9/13)
%
% This template has been downloaded from:
% http://www.LaTeXTemplates.com
%
% Original author:
% Frits Wenneker (http://www.howtotex.com)
%
% License:
% CC BY-NC-SA 3.0 (http://creativecommons.org/licenses/by-nc-sa/3.0/)
%
%%%%%%%%%%%%%%%%%%%%%%%%%%%%%%%%%%%%%%%%%

%----------------------------------------------------------------------------------------
%	PACKAGES AND OTHER DOCUMENT CONFIGURATIONS
%----------------------------------------------------------------------------------------

\documentclass[twoside]{article}

\usepackage[brazilian]{babel}
\usepackage[utf8]{inputenc}
\usepackage[T1]{fontenc}


%\usepackage{lipsum} % Package to generate dummy text throughout this template

\usepackage[sc]{mathpazo} % Use the Palatino font
\usepackage[T1]{fontenc} % Use 8-bit encoding that has 256 glyphs
\linespread{1.05} % Line spacing - Palatino needs more space between lines
\usepackage{microtype} % Slightly tweak font spacing for aesthetics

\usepackage[hmarginratio=1:1,top=32mm,columnsep=20pt]{geometry} % Document margins
\usepackage{multicol} % Used for the two-column layout of the document
\usepackage[hang, small,labelfont=bf,up,textfont=it,up]{caption} % Custom captions under/above floats in tables or figures
\usepackage{booktabs} % Horizontal rules in tables
\usepackage{float} % Required for tables and figures in the multi-column environment - they need to be placed in specific locations with the [H] (e.g. \begin{table}[H])
\usepackage{hyperref} % For hyperlinks in the PDF

\usepackage{lettrine} % The lettrine is the first enlarged letter at the beginning of the text
\usepackage{paralist} % Used for the compactitem environment which makes bullet points with less space between them

\usepackage{abstract} % Allows abstract customization
\renewcommand{\abstractnamefont}{\normalfont\bfseries} % Set the "Abstract" text to bold
\renewcommand{\abstracttextfont}{\normalfont\small\itshape} % Set the abstract itself to small italic text

\usepackage{titlesec} % Allows customization of titles
\renewcommand\thesection{\Roman{section}} % Roman numerals for the sections
\renewcommand\thesubsection{\Roman{subsection}} % Roman numerals for subsections
\titleformat{\section}[block]{\large\scshape\centering}{\thesection.}{1em}{} % Change the look of the section titles
\titleformat{\subsection}[block]{\large}{\thesubsection.}{1em}{} % Change the look of the section titles

\usepackage{fancyhdr} % Headers and footers
\pagestyle{fancy} % All pages have headers and footers
\fancyhead{} % Blank out the default header
\fancyfoot{} % Blank out the default footer
\fancyhead[C]{PUC-Rio - Relatório de INF1631 - Estruturas Discretas $\bullet$ Junho 2014} % Custom header text
\fancyfoot[RO,LE]{\thepage} % Custom footer text

\usepackage{color}
\usepackage{amsmath}
\usepackage{amssymb}
\usepackage{listings}

\definecolor{dkgreen}{rgb}{0,0.6,0}
\definecolor{gray}{rgb}{0.5,0.5,0.5}
\definecolor{mauve}{rgb}{0.58,0,0.82}
\lstset{frame=tb,
  language=C,
  aboveskip=3mm,
  belowskip=0mm,
  belowcaptionskip=0mm,
  numbers=left,
  showstringspaces=false,
  columns=fullflexible,
  basicstyle={\small\ttfamily},
  numbers=none,
  numberstyle=\tiny\color{gray},
  keywordstyle=\color{blue},
  commentstyle=\color{dkgreen},
  stringstyle=\color{mauve},
  breaklines=true,
  breakatwhitespace=true
  tabsize=2,
}

%----------------------------------------------------------------------------------------
%	TITLE SECTION
%----------------------------------------------------------------------------------------

\title{\vspace{-15mm}\fontsize{24pt}{10pt}\selectfont\textbf{Prova, Implementação e Análise de Problemas Selecionados II}} % Article title

\author{
\large
\textsc{Carlos Mattoso, Ian Albuquerque e Leonardo Kaplan}\\[2mm] % Your name
\normalsize PUC-Rio \\ % Your institution
}
\date{}

%----------------------------------------------------------------------------------------

\begin{document}

\maketitle % Insert title

\thispagestyle{fancy} % All pages have headers and footers


%----------------------------------------------------------------------------------------
%	ARTICLE CONTENTS
%----------------------------------------------------------------------------------------

\begin{multicols}{2} % Two-column layout throughout the main article text

\section{Introdução}

\lettrine[nindent=0em,lines=3]{A}presentou-se dois problemas com o objetivo de desenvolver-se códigos com diferentes estruturas de dados e algoritmos para solucioná-los, assim como analisar-se o desempenho das implementações destes algoritmos com respeito ao tempo de CPU.\\

É possível enunciar-se um teorema correspondente para cada um dos problemas descritos, o que permite descrever-los, demonstrar seu entendimento em detalhe, explicar sua fundamentação e justificar sua corretude através de uma prova por indução matemática, montar um algoritmo com base neles e apresentar dados a respeito do tempo de execução dos algoritmos para diversas instâncias. Dessa forma, discriminamos para cada um dos três problemas seus teoremas correspondentes, cada um com seu enunciado, prova,  algoritmo resultate, comentários e resultados.\\

As soluções apresentadas utilizam principalmente a indução, o reforço de hipótese e a técnica conhecida como preenchimento de tabelas ou programação dinâmica.
%------------------------------------------------
\newpage
\section{Problema 1}

\subsection{ Enunciado }
\indent  $\mathbf{Teorema}$ $\mathbf{1}$ : Sabe-se encontrar a árvore geradora máxima de um grafo.

\subsection{ Prova }
\indent A prova é feita por indução matemática utilizando o número de vértices como parâmetro de indução. O \textit{teorema 1} pode ser enunciado:\\

\textit{\textbf{Teorema 1 (k):}} Sabe-se encontrar a árvore de peso máximo de $G=(V,E)$ que contém o vértice $1$ e possui $K$ vértices.\\

\textbf{Teorema do Caso Base:}\\

\textit{\textbf{Teorema 1 (1):}}\\
Existe apenas um vértice $v_1$, portanto ele compõe unicamente a árvore.
Àrvore geradora máxima: $V'_1 = {v_1}$, $E'_1 = \emptyset$.
Peso da árvore geradora máxima: 0\\

\textbf{Teorema do Passo Indutivo:}\\

\textit{\textbf{Teorema 1 (k) $\rightarrow$ Teorema 1 (k+1)}}\\
Por hipótese indutiva conheço $V'_k$ e $E'_k$.\\
Se considerar a componente conexa A formada por $V'_k$ e $E'_k$ e o subgrafo induzido B formado pelos
vértices ($V - V'_k$), necessariamente a aresta de maior peso entre A e B estará na árvore
geradora máxima.\\

$E'_(k+1) = E'_k \cup {argmax(Gama^+(V'_k))}$\\
$V'_(k+1) = V'_k \cup {vértice que liga a aresta que foi adicionada}$\\

\subsection{ Algoritmo Resultante }

%------------------------------------------------
\newpage
\section{Problema 1 - Desafio}

\subsection{ Enunciado }
\indent $\mathbf{Teorema}$ $\mathbf{2}$ Sabe-se encontrar a floresta de peso mínimo de G = (V, E) onde os com-
ponentes conexos possuem pelo menos K vértices.


\subsection{ Prova }
\indent 
\textbf{Teorema(s) do(s) Caso(s) Base(s):}\\
\textbf{Teorema do Passo Indutivo:}\\
\subsection{ Algoritmo Resultante }
\subsection{ Comentários }

%-----------------------------------------------------------------------
\newpage
\section{Problema 2}

\subsection{ Enunciado }
\indent Seja um tabuleiro de xadrez $n$ por $m$ e um rei que está inicialmente na posição (1,1). Para cada posição do tabuleiro estão associados um prêmio $p_ij$ e um consumo $q_ij$. Os prêmios e os consumos assumem somente valores positivos.
O rei tem inicialmente Q unidades para consumir e pode passar quantas vezes quiser em cada
posição do tabuleiro e a cada vez receber o prêmio e, naturalmente, consumir os seus recursos. Ao final (do passeio) o rei tem que estar de volta na posição (1, 1).\\

$\mathbf{Teorema}$ $\mathbf{3}$$\mathbf{(i,j,q)}$ Sabe-se determinar o prêmio máximo que o rei consegue coletar estando na posição (i,j) com q unidades restantes para consumir.\\

\subsection{ Prova }
Para obtermos um reforço de hipótese, reescrevemos da seguinte forma o teorema:

$\mathbf{Teorema}$ $\mathbf{3}$$\mathbf{(x,y,q)}$ Sabe-se determinar o caminho de prêmio máximo que o rei consegue coletar estando na posição (x,y) com q unidades restantes para consumir.\\

\indent 

\textbf{Teorema(s) do(s) Caso(s) Base(s):}\\

\textit{\textbf{Teorema 3 (0):}}\\
Se $ (x,y)=(1,1) \rightarrow P_{max}(x,y,q) = 0$ e $caminho = \emptyset $\\
Caso contrário $\rightarrow P_{max}(x,y,q) = -\infty $ e $caminho = \nexists $ pois não há como chegar em (1,1) \\ 

\textbf{Teorema do Passo Indutivo:}\\

\textit{\textbf{Teorema 3 (x,y,k) $\forall k < q \rightarrow$ Teorema 3 (k+1)}}\\
$V = Vizinhos = \{(i-1,j-1),(i,j-1),(i+1,j-1),(i-1,j),(i+1,j),(i-1,j+1),(i,j+1),(i,j+1),(i+1,j+1)\}$\\
$P_{max}(x,y,q) = max_{v \in V}\{Prêmio(v) + P_{max}(v,q-custo(v))\}$\\

\subsection{ Algoritmo Resultante }



%-----------------------------------------------------------------------
\newpage
\section{Conclusão}
\indent Todos os algoritmos resultantes das provas por indução conseguiram resolver os problemas apresentados. Contudo, para valores de seus respectivos parâmetros que produzam grandes resultados, todos os algoritmos acabaram pecando em termos de desempenho, principalmente em uso de memória.\\



%-----------------------------------------------------
%\section{Results}

%\begin{table}[H]
%\caption{BST}
%\centering
%\begin{tabular}{llr}
%\toprule
%\multicolumn{2}{c}{BST} \\
%\cmidrule(r){2}
%Inserção & Remoção & Busca \\
%\cmidrule(r){2}
%Média & Total & Média & Total & Média & Total\\
%\midrule
%John & Doe & $7.5$ & Richard & Miles & $2$ \\
%\bottomrule
%\end{tabular}
%\end{table}

%\lipsum[5] % Dummy text

%\begin{equation}
%\label{eq:emc}
%e = mc^2
%\end{equation}

%\lipsum[6] % Dummy text

%------------------------------------------------

%\section{Discussion}

%\subsection{Subsection One}

%\lipsum[7] % Dummy text

%\subsection{Subsection Two}

%\lipsum[8] % Dummy text

%----------------------------------------------------------------------------------------
%	REFERENCE LIST
%----------------------------------------------------------------------------------------

%\begin{thebibliography}{99} % Bibliography - this is intentionally simple in this template

%\bibitem[Figueredo and Wolf, 2009]{Figueredo:2009dg}
%Figueredo, A.~J. and Wolf, P. S.~A. (2009).
%\newblock Assortative pairing and life history strategy - a cross-cultural
%  study.
%\newblock {\em Human Nature}, 20:317--330.
 
%\end{thebibliography}

%----------------------------------------------------------------------------------------

\end{multicols}

\end{document}